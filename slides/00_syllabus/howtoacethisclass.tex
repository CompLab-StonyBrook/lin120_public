\documentclass[professionalfonts, xcolor={usenames,svgnames,x11names,table}]{beamer}

\usetheme{SBUclass}
\usepackage[charter]{mathdesign}
\usepackage[scaled]{helvet}

\usepackage{mypackages}
\newcommand{\citecolor}[1]{{\color{gray!85}{#1}}}
\newcommand{\tsb}[1]{\ensuremath{_\text{#1}}}

\title{\texorpdfstring{Language \& Technology}{Language and Technology}}
\subtitle{How to Ace This Class (and University in General)}
\author{Thomas Graf}
\institute{Stony Brook University\\\texttt{lin120@thomasgraf.net}}
\date{}


\begin{document}
\unnumbered{
\begin{frame}
	\titlepage
\end{frame}
}

\begin{frame}{For the Impatient\ldots}
    \begin{itemize}
        \item Do all the exercises.
        \item Spread out exercises and projects rather than doing everything\\
            the evening before the deadline.
        \item Don't fall behind: minor gaps will quickly grow\\
            until you don't understand a word of what's going on.
        \item Be proactive: Play around with Python,\\
            trial-and-error your way to success.
        \item Don't panic if something doesn't work (Python, Blackboard, \ldots).
    \end{itemize}

    Everybody else, please read on.
\end{frame}

\begin{frame}{General Mantras}
    \begin{flushright}
        \begin{minipage}[b]{28em}
            \flushright
            \emph{The mind is not a vessel to be filled but a fire to be kindled.}\\
            Plutarch\\

            \medskip
            \emph{Learning is exploring.}\\
            Yours truly
        \end{minipage}
    \end{flushright}

    \begin{block}{The Gist}
        \begin{itemize}
            \item Be active, not passive.
            \item Develop a thirst for knowledge.
            \item Learn how to learn on your own.
            \item Understanding trumps memorization.
            \item Autonomous thinking is the road to real knowledge.
        \end{itemize}
    \end{block}
\end{frame}

\begin{frame}{Embrace Your Stupidity}
    \begin{quotation}
        [\ldots]
        I actively seek out new opportunities to feel stupid.
        I wouldn't know what to do without that feeling.
        [\ldots]
        The more comfortable we become with being stupid, the deeper we will wade into the unknown and the more likely we are to make big discoveries.
    \end{quotation}
    \begin{flushright}
        Martin A.\ Schwartz\\
        \href{http://jcs.biologists.org/content/121/11/1771}{The importance of stupidity in scientific research}
    \end{flushright}
\end{frame}

\begin{frame}{Keep a Cool Head}
    \begin{itemize}
        \item \textbf{Never panic}\\
            If something doesn't work (e.g.\ your Python code) or\\
            something unexpected happens (e.g.\ Blackboard misbehaving),
            don't panic.
            Panicking only makes things worse.\\
            Keep your emotions in check and see things through rationally.

        \item \textbf{Don't rush into things}\\
            Before you start working on an exercise\slash project, stop for a second.
            \begin{itemize}
              \item Did you fully process what you are supposed to do?
              \item What exactly is the exercise about?
              \item Are there complicating factors?
              \item Does some of the wording seem vague or unclear, and if so, is that a problem with the wording or with your understanding of it?
            \end{itemize}
    \end{itemize}
\end{frame}

\begin{frame}{Be Mindful}
    \begin{itemize}
        \item \textbf{Work on your time management}\\
            \begin{itemize}
                \item Life is messy, in particular college life (until you get bored of all the booze, parties and drama).
                      But that makes good time management even more important.
                \item Familiarize yourself with techniques like GTD. 
                \item Leaving everything to the last minute is not only stressful,\\
                    it also impedes your learning progress.
            \end{itemize}

        \item \textbf{Plan ahead and pace yourself}\\
            \begin{itemize}
                \item Spread out exercises over multiple days.
                \item The amount of time you invest is not as important for learning as how that time is distributed.
                      30 minutes a day for 4 days is more efficient than a single 2 hour session. 
            \end{itemize}

        \item \textbf{Reflect on what you are doing}\\
            \begin{itemize}
                \item Don't mindlessly do X just because X was assigned in class.
                \item Ask yourself why it is a requirement, what skills it is supposed to teach you, and how that gels with your own goals and expectations. 
            \end{itemize}
    \end{itemize}
\end{frame}

\begin{frame}{Ask for Help}
    \begin{itemize}
        \item \textbf{Take advantage of us}\\
            We put a lot of effort into helping you achieve your goals:
            %
            \begin{itemize}
                \item recitations
                \item office hours
                \item availability via email and Google hangouts
            \end{itemize}
            %
            If you don't take advantage of these opportunities,\\
            you have no right to complain about grades or homeworks. 
            
        \item \textbf{Take advantage of each other}\\
            Your peers are a valuable resource, too.
            Discuss homeworks, exchange ideas, share notes.
            Collaborate, help each other.

        \item \textbf{Don't wait too long}\\
            The Matthew effect also applies to education: the rich get richer, the poor get poorer.
            If you sense yourself falling behind, ask for help right away.
            The longer you wait, the worse it gets.
    \end{itemize}
\end{frame}

\begin{frame}{Have Fun}
    \begin{itemize}
        \item \textbf{Be playful}\\
            Whatever the subject matter, play around with it.
            \begin{itemize}
                \item Learning Python? Then open up the shell once in a while and just write a small program to play three-dimensional Tic-Tac-Toe.
                \item Memorizing US history facts? Let's see if we can condense them into a little poem (like the presidents song by the \emph{Animaniacs}).
                \item Studying protein folding? Fix yourself a serving of spaghetti and try some common folding patterns.
            \end{itemize}

        \item \textbf{Hang out like the Vienna Circle}\\
            Make learning a social event.
            Meet with your peer for coffee\slash drinks to discuss the class.
            Explain specific points to each other --- teaching others benefits your own learning.
    \end{itemize}
\end{frame}

\begin{frame}{Finally, a Dirty Secret}
    \highlight{Grades don't matter much!}

    \medskip
    \begin{itemize}
        \item Unless you want to go to med-\slash law-\slash grad-school or need a stipend, forget grades.
              The real world doesn't care about your grades.
        \item A grade is just our feedback to you how much you know\\
            about a given topic.
            It's your choice what you do with it.
        \item A bad grade does not ruin your life or make you a failure,\\
            just like a high GPA is no guarantee for a successful career.
        \item But this is no license to slack off!
            \begin{itemize}
                \item A degree opens doors, but that won't last if you suck at your job.
                \item College life gives you a few precious years to prepare yourself for the real world.
                    Don't squander them.
            \end{itemize}
    \end{itemize}
\end{frame}

\end{document}
